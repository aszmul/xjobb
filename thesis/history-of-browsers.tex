\chapter{History of the Internet and browsers}
  \begin{metatext}
    Browsers and the Internet is something that many people today take for granted.
    It is not longer the case that only computer scientists are browsing the \gls{web}.
    Today the \gls{web} is becoming increasingly important in both our personal and professional lives.
    This chapter will give a brief history of \glspl{browser} and how the \gls{web} transitioned from handling science \glspl{document} to commercial applications.
    This section is a summary of \normalfont{\cite{internet_live_stats,internet_of_things,wiki_hypermedia,w3c_www,oed}}.
  \end{metatext}

  \todo[inline]{Change all wikipedia sources to the "real" sources?}

  Before addressing the birth of the \gls{web}, it is necessary to define the meaning of the \emph{Internet} and the \emph{\gls{WWW}}.
  The word ``internet'' can be translated to \emph{something between networks}.
  When referring to the Internet (capitalized) it is usually the global decentralized internet used for communication between millions of networks using the \gls{TCP} and \gls{IP} suite.
  Since the Internet is decentralized, there is no single owner of the network.
  In other words, the owners are all the network end-points (all users of the Internet).
  One can argue that the owners of the Internet are the \glspl{ISP}, providing the services and infrastructure making the Internet possible.
  On the other hand, the backbones of the Internet are usually co-founded nationally.
  \todo[color=green]{Marcos: s/co-founded/controlled ?}
  Also, it is the \gls{ICANN} organization that has the responsibility for managing the \gls{IP} addresses in the Internet namespace, which reduces the ownership of the \glspl{ISP} further.
  Clearly, the Internet wouldn't be what it is today without all the actors.
  The Internet lays the ground for many systems and applications, including the \gls{WWW}, file sharing and telephony.
  In 2014 the number of Internet users was measured to just below 3 billions, and estimations show that we have surpassed 3 billions users today (no report for 2015 has been published yet).
  Users are defined as humans having access to the Internet at home.
  If one instead measures the number of connected entities (electronic devices that communicates through the Internet) the numbers are much higher.
  An estimation for 2015 of 25 billions connected entities has been made, and the estimation for 2020 is 50 billion.
  
  As already stated, the \gls{WWW} is a system that operates on top of the Internet.
  The \gls{WWW} is usually shortened to simply \emph{the \gls{web}}.
  The \gls{web} is an information space that interoperates through standardized protocols and standards, which affords users with the ability to access various types of resources.
  This can include interlinked \gls{hypertext} \glspl{document}, which themselves can contain other media such as images and videos, and/or data services.
  Since not only \gls{hypertext} is interlinked on the \gls{web}, the term \emph{\glslink{hypertext}{hypermedia}} can be used as an extension to \gls{hypertext} that also includes other nonlinear medium of information.
  Although the term \glslink{hypertext}{hypermedia} has been around for a long time, the term \gls{hypertext} is still being used as a synonym for \glslink{hypertext}{hypermedia}.
  Further, the \gls{web} can also be referred to as the universe of information accessible through the \gls{web} system.
  Therefore, the \gls{web} is both the system enabling sharing of \glslink{hypertext}{hypermedia} and also all of the accessible \glslink{hypertext}{hypermedia} itself.
  Hypertext \glspl{document} are today more known by the name \emph{\gls{web} pages} or simply \emph{pages}.
  \todo[color=green]{Marcos: Maybe drop the term pages since it is a bit outdated?}
  Multiple related pages (that are often served from the same domain) compose a \emph{web site} or simply a \emph{site}.
  Hypertext \glspl{document} are written in \gls{HTML}, and often includes \gls{CSS} for styling and \gls{JavaScript} for custom user interactions.
  To transfer the resources between computers the protocol \gls{HTTP} is used.
  Typically the way of retrieving resources on the web is by using a \emph{\glslink{browser}{user agent}}, known colloquially as a \gls{web} \emph{\gls{browser}}.
  Typically the way of retrieving resources on the \gls{web} is by using a \emph{\gls{web} \gls{browser}} or simply a \emph{\gls{browser}}.
  Browsers handle the fetching, parsing and rendering of the \gls{hypertext} (more about this in section~\ref{sec:browsers}).
  
  \section{The history of the Internet}\label{sec:history-internet}
    \begin{metatext}
      Since the \gls{web} is a system operating on top of the Internet, it is needed to first investigate the history of the Internet.
      This can be viewed from many angles and different aspects need to be taken into consideration.
      With that in mind, the origin of the Internet is not something easily pinned down and what will be presented here will be more technically interesting than the exact history.
      This section is a summary of \normalfont{\cite{overview_of_tcp_ip,internetsociety_history_internet,internet_maps,historyofthings_internet}}.
    \end{metatext}

    \todo[inline,color=green]{Marcos: This doesn't relate to your thesis at all, TBH - which is about layout problems. I would remove this unless it somehow relates to layout of information.}

    In the early 1960's \emph{packet switching} was being researched, which is a prerequisite of internetworking.
    With packet switching in place, the very important ancestor of the Internet \gls{ARPANET} was developed, which was the first network to implement the \gls{TCP}/\gls{IP} suite.
    The \gls{TCP}/\gls{IP} suite together with packet switching are fundamental technologies of the Internet.
    \gls{ARPANET} was funded by the \gls{US} \gls{DOD} in order to interconnect their research sites in the \gls{US}.
    The first nodes of \gls{ARPANET} was installed at four major universities in the western \gls{US} in 1969 and two years later the network spanned the whole country.
    The first public demonstration of \gls{ARPANET} was held at the \gls{ICCC} in 1972.
    It was also at this time the email system was introduced, which became the largest network application for over a decade.
    In 1973 the network had international connections to Norway and London via a satellite link.
    At this time information was exchanged with the \gls{FTP}, which is a protocol to transfer files between hosts.
    This can be viewed as the first generation of the Internet. With around 40 nodes, operating with raw file transfers between the hosts it was mostly used by the academic community of the \gls{US}.

    The number of nodes and hosts of \gls{ARPANET} increased slowly, mainly due to the fact that it was a centralized network owned and operated by the \gls{US} military.
    In 1974 the \gls{TCP}/\gls{IP} suite was proposed in order to have a more robust and scalable system for end-to-end network communication.
    The \gls{TCP}/\gls{IP} suite is a key technology for the decentralization of the \gls{ARPANET}, which allowed the massive expansion of the network that later happened.
    In 1983 \gls{ARPANET} switched to the \gls{TCP}/\gls{IP} protocols, and the network was split in two.
    One network was still called \gls{ARPANET} and was to be used for research and development sites.
    The other network was called \gls{MILNET} and was used for military purposes.
    The decentralization event was a key point and perhaps the birth of the Internet.
    The \gls{CSNET} was funded by the \gls{NSF} in 1981 to allow networking benefits to academic intsitutions that could not directly connect to \gls{ARPANET}.
    After the event of decentralizing \gls{ARPANET}, the two networks were connected among many other networks.
    In 1985 \gls{NSF} started the \gls{NSFNET} program to promote advanced research and education networking in the \gls{US}.
    To link the supercomputing centers funded by \gls{NSF} the \gls{NSFNET} served as a high speed and long distance backbone network.
    As more networks and sites were linked by the \gls{NSFNET} network, it became the first backbone of the Internet.
    In 1992, around 6000 networks were connected to the \gls{NSFNET} backbone with many international networks.
    To this point, the Internet was still a network for scientists, academic institutions and technology enthusiasts.
    Mainly because \gls{NSF} had stated that \gls{NSFNET} was a network for non-commercial traffic only.
    In 1993 \gls{NSF} decided to go back to funding research in supercomputing and high-speed communications instead of funding and running the Internet backbone.
    That, along with an increasing pressure of commercializing the Internet let to another key event in the history of the Internet - the privatization of the \gls{NSFNET} backbone.

    In 1994, the \gls{NSFNET} was systematically privatized while making sure that no actor owned too much of the backbone in order to create constructive market competition.
    With the Internet decentralized and privatized regular people started using it as well as companies.
    Backbones were built across the globe, more international actors and organizations appeared and eventually the Internet as we know it today came to exist.

  \section{The birth of the World Wide Web}\label{sec:www}
    \begin{metatext}
      Now that the history of the Internet has been described, it is time to talk about the birth of the web.
      Here the initial ideas of the \gls{web} will be described, the alternatives and how it became a global standard.
      This subsection is a summary of \normalfont{\cite{wiki_gopher,wiki_www,webdevnotes_history_of_the_internet,webdevnotes_www_basics,historyofthings_internet}}.
    \end{metatext}

    \todo[inline,color=green]{Marcos: Same with this. This has been described many times, I would recommend dropping this as it adds little value and doesn't relate to layout problems.}

    \noindent
    Recall from section~\ref{sec:history-internet} that the way of exchanging information was to upload and download files between clients and hosts with \gls{FTP}.
    If a \gls{document} downloaded was referring to another \gls{document}, the user had to manually find the server that hosted the other \gls{document} and download it manually.
    This was a poor way of digesting information and \glspl{document} that linked to other resources.
    In 1989 a proposal for a communication system that allowed interlinked \glspl{document} was submitted to the management at \gls{CERN}.
    The idea was to allow links to other \glspl{document} embedded in text \glspl{document}, directly accesible for users.
    A quote from the draft:
    \begin{quote}
      Imagine, then, the references in this \gls{document} all being associated with the network address of the thing to which they referred, so that while reading this \gls{document} you could skip to them with a click of the mouse.
    \end{quote}
    This catches the whole essence of the \gls{web} in a sentence --- to interlink resources in an user friendly way.
    The proposal describes that such text embedded links would be \gls{hypertext}.
    It continues to explain that interlinked resources does not need to be limited to text \glspl{document} since multimedia such as images and videos can also be interlinked which would similarly be hypermedia.
    The concept of \glspl{browser} is described, with a client-server model the \gls{browser} would fetch the \gls{hypertext} \glspl{document}, parse them and handle the fetching of all media linked in the \gls{hypertext}.

    In 1990, \gls{HTTP} and \gls{HTML} were implemented by Tim Bernes Lee at \gls{CERN}.
    A \gls{browser} and a \gls{web} server were also created and the \gls{web} was born.
    One year later the \gls{web} was introduced to the public and in 1993 over five hundred international \gls{web} servers existed.
    It was stated in 1994 by \gls{CERN} that the \gls{web} was to be free without any patents or royalties.
    \todo[color=green]{Marcos: Citation needed.}
    At this time the \gls{W3C} was founded with support from the \gls{DARPA} and the European Commission.
    \todo[color=green]{Marcos: citation needed.}
    The organization comprised of companies and individuals that wanted to standardize and improve the \gls{web}.

    As a side note, the Gopher protocol was developed in parallel to the web by the University of Minnesota.
    It was released in 1991 and quickly gained traction as the \gls{web} still was in very early stages.
    The goal of the system, just like the \gls{web}, was to overcome the shortcomings of browsing \glspl{document} with \gls{FTP}.
    Gopher enabled servers to list the \glspl{document} present, and also to link to \glspl{document} on other servers.
    This created a strong hierarchy between the \glspl{document}.
    The listed \glspl{document} of a server could then be presented as \gls{hypertext} menus to the client (much like a \gls{web} \gls{browser}).
    As the protocol was simpler than \gls{HTTP} it was often preferred since it used less network resources.
    The structure provided by Gopher provided a platform for large electronic library connections.
    A big difference between the \gls{web} and the Gopher platform is that the Gopher platform provided \gls{hypertext} menus presented as a file system while the \gls{web} \gls{hypertext} links inside \gls{hypertext} \glspl{document}, which provided greater flexibility.
    When the University of Minnesota announced that it would charge licensing fees for the implementation, users were somewhat scared away.
    As the \gls{web} matured, being a more flexible system with more features as well as being totally free it quickly became dominant.

  \section{The history of browsers}
    \label{sec:browsers}
    \begin{metatext}
      In the mid 1990's the usage of the Internet transitioned from downloading files with \gls{FTP} to instead access resources with the \gls{HTTP} protocol.
      To fulfill the vision that users would be able to skip to the linked \glspl{document} ``with a click of the mouse'' users needed a client to handle the fetching and displaying of the \gls{hypertext} \glspl{document}, hence the need for \glspl{browser} were apparent.
      Here the evolution of the \gls{browser} clients will be given, while emphasizing the timeline of the popular \glspl{browser} we use today.
      This section is a summary of \normalfont{\cite{wiki_www,tim_wiki,kesan2003deconstructing,sink2003memoirs,wiki_mozilla,wiki_opera,wiki_konqueror,wiki_safari,wiki_webkit,wiki_blink}}.
    \end{metatext}

    \todo[inline,color=green]{Marcos: This only gets interesting (with relation to the thesis) when you start talking about layout. The rest of the history is not relevant to the thesis.}

    The first \gls{web} \gls{browser} ever made was created in 1990 and was called WorldWideWeb (which was renamed to Nexus to avoid confusion).
    \todo[color=green]{Marcos: Confusion with what?}
    It was at the time the only way to view the \gls{web}, and the \gls{browser} only worked on NeXT computers.
    Built with the NeXT framework, it was quite sophisticated.
    It had a \gls{GUI} and a \gls{WYSIWYG} \gls{hypertext} \gls{document} editor.
    Unfortunately it couldn't be ported to other platforms, so a new \gls{browser} called \emph{Line Mode Browser} (\abbr{lmb}) was quickly developed.
    \todo[color=green]{Marcos: Developed by who?}
    To ensure compatibility with the earliest computer terminals the \gls{browser} displayed text, and was operated with text input.
    Since the \gls{browser} was operated in the terminal, users could log in to a remote server and use the \gls{browser} via telnet.
    In 1993, the core browser code was extracted and rewritten in C to be bundled as a library called \emph{libwww}.
    The library was licensed as \emph{public domain} to encourage the development of \gls{web} \glspl{browser}.
    Many \glspl{browser} were developed at this time.
    The \emph{Arena} \gls{browser} served as a testbed \gls{browser} and authoring tool for Unix.
    The \emph{ViolaWWW} \gls{browser} was the first to support embedded scriptable objects, stylesheets and tables.
    \emph{Lynx} is a text-based \gls{browser} that supports many protocols (including Gopher and \gls{HTTP}), and is the oldest \gls{browser} still being used and developed.
    The list of \glspl{browser} of this time can be made long.

    In 1993, the \emph{Mosaic} \gls{browser} was released by the \gls{NCSA} which came to be the ancestor of many of the popular \glspl{browser} in use today.
    As Lynx, Mosaic also supported many different protocols.
    Mosaic quickly became popular, mainly due to its intuitive \gls{GUI}, reliability, simple installation and Windows compatibility.
    The company \emph{Spyglass, Inc.} licensed the \gls{browser} from the \gls{NCSA} for producing their own \gls{browser} in 1994.
    Around the same time the leader of the team that developed Mosaic, Marc Andreessen, left the \gls{NCSA} to start \emph{Mosaic Communications Corporation}.
    The company released their own \gls{browser} named \emph{Mosaic Netscape} in 1994, which later was to be called \emph{Netscape Navigator} that was internally codenamed \emph{Mozilla}.
    Microsoft licensed the Spyglass Mosaic \gls{browser} in 1995, modified and renamed it to \emph{Internet Explorer}.
    In 1997 Microsoft started using their own \emph{Trident} \gls{layout engine} for Internet Explorer.
    The Norwegian telecommunications company \emph{Telenor} developed their own \gls{browser} called \emph{Opera} in 1994, which was released 1996.
    Internet Explorer and Netscape Navigator were the two main \glspl{browser} for many years, competing for market dominance.
    Netscape couldn't keep up with Microsoft, and was slowly losing market share.
    In 1998 Netscape started the open source Mozilla project, which made available the source code for their \gls{browser}.
    Mozilla was to originally develop a suite of Internet applications, but later switched focus to the \emph{Firefox} \gls{browser} that had been created in 2002.
    Firefox uses the \emph{Gecko} \gls{layout engine} developed by Mozilla.

    Another historically important \gls{browser} is the \emph{Konqueror} \gls{browser} developed by the free software community \gls{KDE}.
    The \gls{browser} was released in 1998 and was bundled in the \gls{KDE} Software Compilation.
    Konqueror used the \abbr{khtml} \gls{layout engine}, also developed by \gls{KDE}.
    In 2001, when \emph{Apple Inc.} decided to build their own \gls{browser} to ship with \abbr{os x}, a \gls{fork} called \gls{WebKit} was made of the \abbr{khtml} project.
    Apple's \gls{browser} called \emph{Safari} was released in 2003.
    The \gls{WebKit} \gls{layout engine} was made fully open source in 2005.
    In 2008, \emph{Google Inc.} also released a \gls{browser} based on \gls{WebKit}, named \emph{Chrome}.
    The majority of the source code for Chrome was open sourced as the \emph{Chromium} project.
    Google decided in 2013 to create a \gls{fork} of \gls{WebKit} called \emph{Blink} for their \gls{browser}.
    Opera Software decided in 2013 to base their new version of Opera on the Chromium project, using the Blink \gls{fork}.