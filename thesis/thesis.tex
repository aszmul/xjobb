\documentclass[a4paper,11pt]{kth-mag}
\usepackage[T1]{fontenc}
\usepackage{textcomp}
\usepackage{lmodern}
\usepackage[utf8]{inputenc}
\usepackage{csquotes}
\usepackage[swedish,english]{babel}
\usepackage{modifications}
\usepackage[backend=biber]{biblatex}

\bibliography{bibliography.bib}
\title{Modular responsive web design}
\foreigntitle{Modulär responsiv webbutveckling}
\subtitle{Allowing responsive web modules to respond to custom criterias instead of only viewport size by implementing \emph{element queries}}
\author{Lucas Wiener \\ \lowercase{lwiener@kth.se}}
\date{February 2015}
\blurb{Master's Thesis at \textsc{csc}\\\hfill\\ Supervisors at \textsc{evry ab}: Tomas Ekholm \& Stefan Sennerö\\Supervisor at \textsc{csc}: Philipp Haller\\Examiner: Mads Dam}
\trita{TRITA xxx yyyy-nn}
\begin{document}
  \frontmatter
  \pagestyle{empty}
  \removepagenumbers
  \maketitle
  \selectlanguage{english}
  \begin{abstract}
    Abstract goes here.
  \end{abstract}
  \clearpage
  \begin{foreignabstract}{swedish}
    Sammanfattning ska vara här.
  \end{foreignabstract}
  \clearpage
  \tableofcontents*
  \mainmatter
  \pagestyle{newchap}
  \chapter{Introduction}
    \section{Targeted audience}
    \section{Problem statement}
    \section{Objective}
    \section{Methodology}
    \section{Delimitations}
    \section{Outline}
  \part{Background}

    \chapter{Browsers} 
      Browsers and the Internet is something that many people today take for granted.
      It is not longer the case that only computer scientists are browsing the web.
      Today the web is becoming increasingly important in both our personal and professional lives.
      This chapter will give a brief history of browsers and the rise of the web.
      It will also cover the role of browsers today, and what can be expected in the future. \cite{internetsociety_history_internet}
      \section{The origin of the web}
        Before addressing the birth of the web, lets define the meaning of the the concepets of the \emph{Internet}, \emph{Web} and \emph{World Wide Web}.
        The word internet can be translated to \emph{something between networks}. 
        When referring to \emph{the Internet} (capitalized) it is usually the global decentralized internet used for communication between millions of networks using \textsc{tcp/ip}.
        Since the Internet is decentralized, there is no owner.
        Or in other words, the owners are all the network end-points which means all users of the Internet.
        One can argue that the owners of the Internet are the ISP's, providing the services and infrastructure making the Internet possible.
        On the other hand, the backbones of the Internet are usually co-founded by countries and companies.
        Or is it the ICANN organization which has the responsibility for managing the IP addresses in the Internet namespace?
        Clearly, the Internet wouldn't be what it is today without all the actors.
        The Internet lays the ground for many systems and applications, including the World Wide Web, file sharing and telephony.
        In 2014 the number of Internet users was measured to just below 3 billions, and estimations shows that we have surpassed 3 billions users today (no report for 2015 has been made yet) \cite{internet_live_stats}.
        Users are here defined as humans having unrestricted acccess to the Internet \cite{internet_live_stats}.
        If one instead measures the number of connected entities (electronic devices that communicates through the Internet) the numbers are much higher.
        An estimation for 2015 of 25 billions connected entities has been made, and the estimation for 2020 is 50 billions \cite{internet_of_things}.
        
        As already stated, the Word Wide Web (abbreviated WWW or W3) is 

    \chapter{Web development}
      \section{From documents to applications}
      \section{Responsive web design}
    \chapter{Modular development}
      \section{Web Components}
  \part{Theory}
    \chapter{Rendering engines}
    \chapter{Element queries}
  \part{Third-party framework}
    \chapter{Analysis of approaches}
    \section{Current implementations}
    \chapter{API design}
    \chapter{Implementation}
  \part{Result}
    \chapter{Discussion}
  \printbibliography
  \appendix
  \addappheadtotoc
  \chapter{RDF}\label{appA}
    \begin{figure}[ht]
      \begin{center}
        And here is a figure
        \caption{\small{Several statements describing the same resource.}}\label{RDF_4}
      \end{center}
    \end{figure}
  that we refer to here: \ref{RDF_4}
\end{document}
