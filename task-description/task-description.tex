\documentclass[oneside,a4paper,11pt]{kth-mag}
\usepackage[T1]{fontenc}
\usepackage{textcomp}
\usepackage{lmodern}
\usepackage[latin1]{inputenc}
\usepackage[swedish,english]{babel}
\usepackage{modifications}
\title{Modular responsive web design}

\subtitle{Allowing a single component to be dropped into any layout and have them respond appropriately}
\foreigntitle{Allowing a single component to be dropped into any layout and have them respond appropriately}
\author{Lucas Wiener}
\date{December 2014}
\blurb{Master's Thesis task description}
\trita{}
\begin{document}
\frontmatter
\pagestyle{empty}
\removepagenumbers
\maketitle
\mainmatter
\section*{Introduction}

Many good developers agree that modular development is an important factor of successful software development. To break systems up in modules and having these modules be responsible and take care of well defined tasks enable programmers to develop their products in a more controlled and reliable way. Solving problems in the context of a module with a single responsibility is simpler than solving problems in a context which also does many other things. Furthermore, modules are easy to reuse in many different projects due to the single responsibility nature of modules. Again, many developers have seen the benefits of using modules and as of writing this text there are millions of modules (also called plugins, libraries, components, etc.) online for the world to use. Most development environments enable developers to work in a modular way and usually encourages it. Web programmers are used to using third-party libraries and components to build complex web apps with modular interfaces (i.e. views that consists of smaller modular views).

With the rise of smartphones and tablets, web developers realized that they needed a good way of having their websites adapt to the different devices. The solution came to be responsive web design, which is a technique to make websites react and adapt to the screen size of the viewport, so that the website elements are displayed in a customized way for different viewport sizes. This enables big desktop websites to shrink to smaller and mobile friendly versions when the browser window is resized. This was a huge breakthrough in web development and responsive web design is more popular than ever. The idea of having components and modules morph to more suitable designs given a size is amazing. It is just too bad that the current implementation of responsive web design only takes the viewport into consideration. This limits web developers to build truly modular and responsive interfaces.

Imagine a website displaying a news feed. Each post of the news feed contains a lot of information; title, body, location, time, source, author, etc. The app is responsive, so that when the news feed is viewed on a small screen (less than 600 pixels wide for example) only the important information is shown to the user; title and body. The news feed is a module, so it can easily be integrated into other web apps. But what if another web app just want to have the smaller version of the news feed? If the web app put the news feed module in a container less than 600 pixels wide the news feed would react to the small width and adapt itself to the smaller version, right? Unfortunately not. This is the issue I would like to address in my master thesis. Writing responsive modules that can adapt to parent element sizes instead of viewport sizes is a natural step forward in responsive web design. This way web developers would truly be able to build and use modular web components.

\end{document}
